%-----------------------------------------------------------------------------------------------------------------------------------------------%
%	The MIT License (MIT)
%
%	Copyright (c) 2021 Jitin Nair
%
%	Permission is hereby granted, free of charge, to any person obtaining a copy
%	of this software and associated documentation files (the "Software"), to deal
%	in the Software without restriction, including without limitation the rights
%	to use, copy, modify, merge, publish, distribute, sublicense, and/or sell
%	copies of the Software, and to permit persons to whom the Software is
%	furnished to do so, subject to the following conditions:
%	
%	THE SOFTWARE IS PROVIDED "AS IS", WITHOUT WARRANTY OF ANY KIND, EXPRESS OR
%	IMPLIED, INCLUDING BUT NOT LIMITED TO THE WARRANTIES OF MERCHANTABILITY,
%	FITNESS FOR A PARTICULAR PURPOSE AND NONINFRINGEMENT. IN NO EVENT SHALL THE
%	AUTHORS OR COPYRIGHT HOLDERS BE LIABLE FOR ANY CLAIM, DAMAGES OR OTHER
%	LIABILITY, WHETHER IN AN ACTION OF CONTRACT, TORT OR OTHERWISE, ARISING FROM,
%	OUT OF OR IN CONNECTION WITH THE SOFTWARE OR THE USE OR OTHER DEALINGS IN
%	THE SOFTWARE.
%	
%
%-----------------------------------------------------------------------------------------------------------------------------------------------%

%----------------------------------------------------------------------------------------
%	DOCUMENT DEFINITION
%----------------------------------------------------------------------------------------

% article class because we want to fully customize the page and not use a cv template
\documentclass[a4paper,12pt]{article}

%----------------------------------------------------------------------------------------
%	FONT
%----------------------------------------------------------------------------------------

% % fontspec allows you to use TTF/OTF fonts directly
% \usepackage{fontspec}
% \defaultfontfeatures{Ligatures=TeX}

% % modified for ShareLaTeX use
% \setmainfont[
    % SmallCapsFont = Fontin-SmallCaps.otf,
    % BoldFont = Fontin-Bold.otf,
    % ItalicFont = Fontin-Italic.otf
    % ]
    % {Fontin.otf}
    
    %----------------------------------------------------------------------------------------
    %	PACKAGES
    %----------------------------------------------------------------------------------------
    \usepackage{url}
    \usepackage{parskip} 	
    
    %other packages for formatting
    \RequirePackage{color}
    \RequirePackage{graphicx}
    \usepackage[usenames,dvipsnames]{xcolor}
    \usepackage[scale=0.9]{geometry}
    
    %tabularx environment
    \usepackage{tabularx}
    
    %for lists within experience section
    \usepackage{enumitem}
    
    % centered version of 'X' col. type
    \newcolumntype{C}{>{\centering\arraybackslash}X} 
    
    %to prevent spillover of tabular into next pages
    \usepackage{supertabular}
    \usepackage{tabularx}
   % \usepackage{ltablex}
    \newlength{\fullcollw}
    \setlength{\fullcollw}{0.47\textwidth}
    
    %custom \section
    \usepackage{titlesec}				
    \usepackage{multicol}
    \usepackage{multirow}
    \usepackage[catalan]{babel}
    \usepackage{fontawesome5} % Paquet d'icones modern
    \definecolor{linkcolour}{rgb}{0,0.2,0.6}
    
%CV Sections inspired by: 
%http://stefano.italians.nl/archives/26
\titleformat{\section}{\Large\scshape\raggedright\color{linkcolour}}{}{0em}{}[\color{linkcolour}\titlerule]
\titlespacing{\section}{0pt}{10pt}{10pt}

%for publications
\usepackage[style=authoryear,sorting=ynt, maxbibnames=2]{biblatex}

%Setup hyperref package, and colours for links
\usepackage[unicode, draft=false]{hyperref}
\definecolor{linkcolour}{rgb}{0,0.2,0.6}
\hypersetup{colorlinks,breaklinks,urlcolor=linkcolour,linkcolor=linkcolour}
\addbibresource{citations.bib}
\setlength\bibitemsep{1em}

%for social icons
\usepackage{fontawesome5}

%debug page outer frames
%\usepackage{showframe}


% job listing environments
\newenvironment{jobshort}[2]
    {
    \begin{tabularx}{\linewidth}{@{}l X r@{}}
    \textbf{#1} & \hfill &  #2 \\[3.75pt]
    \end{tabularx}
    }
    {
    }

\newenvironment{joblong}[2]
    {
    \begin{tabularx}{\linewidth}{@{}l X r@{}}
    \textbf{#1} & \hfill &  #2 \\[3.75pt]
    \end{tabularx}
    \begin{minipage}[t]{\linewidth}
    \begin{itemize}[nosep,after=\strut, leftmargin=1em, itemsep=3pt,label=--]
    }
    {
    \end{itemize}
    \end{minipage}    
    }



%----------------------------------------------------------------------------------------
%	BEGIN DOCUMENT
%----------------------------------------------------------------------------------------
\begin{document}

% non-numbered pages
\pagestyle{empty} 

%----------------------------------------------------------------------------------------
%	TITLE
%----------------------------------------------------------------------------------------

% \begin{tabularx}{\linewidth}{ @{}X X@{} }
% \huge{Your Name}\vspace{2pt} & \hfill \emoji{incoming-envelope} email@email.com \\
% \raisebox{-0.05\height}\faGithub\ username \ | \
% \raisebox{-0.00\height}\faLinkedin\ username \ | \ \raisebox{-0.05\height}\faGlobe \ mysite.com  & \hfill \emoji{calling} number
% \end{tabularx}

\begin{tabularx}{\linewidth}{@{}l X@{}}
\begin{minipage}[c]{0.20\linewidth}
\includegraphics[width=2.6cm,height=2.6cm,keepaspectratio]{media/ProfilePicture.png}
\end{minipage} &
\begin{minipage}[c]{0.78\linewidth}
{\Huge\bfseries Quim Nadal Bargués}\\[4pt]
{\small
\href{https://www.linkedin.com/in/quimnadalbargues/}{\color{linkcolour}\faLinkedin\ \ Quim Nadal Bargués} \ $|$ 
\href{mailto:quimnaba@gmail.com}{\color{linkcolour}\faEnvelope\ \ quimnaba@gmail.com} \ $|$ \\
\href{tel:+34620200817}{\color{linkcolour}\faMobile\ \ +34 620.200.817} \ $|$ 
\href{https://github.com/quimnaba123}{\color{linkcolour}\faGithub\ \ quimnaba123} \ $|$ 
\href{}{\raisebox{-0.05\height}{\color{linkcolour}\faHome} \textcolor{linkcolour}{Barcelona}}
}
\end{minipage} \\
\end{tabularx}
\vspace{6pt}
\noindent{\color{linkcolour}\rule{\linewidth}{1pt}}\par\vspace{6pt}

%----------------------------------------------------------------------------------------
% EXPERIENCE SECTIONS
%----------------------------------------------------------------------------------------

%Interests/ Keywords/ Summary
\section{Perfil Professional}
Enginyer de Telecomunicacions amb més de 7 anys d'experiència en entorns d'alt rendiment, expert en sistemes de control electrònic, 
aplicacions de software i sistemes incrustats. Sòlida trajectòria dirigint i implementant projectes de software internacionals 
(principalment a Dinamarca) mitjançant metodologies àgils, participant en la definició de requisits, el disseny i la validació, 
amb especial atenció a la seguretat, la monitorització i la gestió d'infraestructures tecnològiques.
Capacitat demostrada per liderar equips tècnics, dissenyar sistemes resilients i optimitzar processos 
tècnics i operatius, assegurant el compliment de polítiques de seguretat i la continuïtat del negoci.

%Experience
\section{Experiència Professional}

\begin{jobshort}{Enginyer de Software | Creadis S.A (Barcelona - Híbrid)}{Novembre 2021 - Actualitat}\\
    Responsable de la creació i el desenvolupament d'aplicacions de software per a sistemes electrònics de control per a sistemes 
    de producció, transport i emmagatzemtage d'energia eòlica, cobrint tot el cicle de vida: 
    definició de requisits, disseny de conceptes a implementar, l'arquitectura de sistema, l'arquitectura de software, la implementació i validació de software, 
    gestió del projecte i del procés àgil i avaluació de riscos (amb especial atenció als requisits de ciberseguretat).
    Coordinació i execució de projectes de processament de dades en temps real 
    en controladors des de més de 200 sensors, inclòs el tractament i la supervisió d'alarmes, la redundància de sistemes i la detecció 
    d'incidències. Recentment, en paral·lel amb el projecte actual, participant en el desenvolupament d'un projecte intern d'optimització de dimensionament 
    d'una planta energètica (bateries, la seva ubicació...) en dependència amb un històric dels preus de mercat de l'energia per tal d'obtenir un màxim de
    benefici econòmic de la compra-venta d'energia.
    En particular, destacar el desenvolupament dels següents projectes amb els següents clients:\\ \\
    \vspace{0.3em}
    \textbf{Vestas Wind Systems A/S}, Disseny i Desenvolupament d'un Controlador d'arrancada d'una Planta de generació d'energia eòlica, \\
    Setembre 2025 – Actualitat, Dinamarca (remot)
    \begin{itemize}
        \item Participació en la definició del projecte d'un controlador de planta d'energia eòlica, que permet l'arrancada intel·ligent de les turbines: 
        Definició de conceptes, disseny dels conceptes, definició dels requisits (unitaris i d'integració) i gestió de riscos.
        \item Participació en la implementació dels requisits en base al disseny, en codi Matlab/Simulink, i en la corresponent validació de 
        la implementació a través de test unitaris fets amb Simulink Test Harness, avaluant el compliment dels requisits.
        \item Participació en la definició de l'estratègia global de l'empresa envers el projecte, avaluant punts febles i riscos, i adaptant la gestió del projecte
        intern al procés d'entrega de productes de software del client  al mercat.
        \item Modularització de la implementació per anticipar futures càrregues de treball en moments crítics del projecte.
    \end{itemize}
    \vspace{0.3em}
    \textbf{Siemens Energy, SA}, Software de Control i Càrregues, \\Gener 2023 - Juliol 2025, Dinamarca (remot)
    \begin{itemize}
        \item Responsable del disseny, implementació i verificació de sistemes de control de càrrega, en Matlab/Simulink, 
        processant dades de sensors a través d'algorismes per obtenir dades d'altres sensors i augmentar la fiabilitat global
        de la turbina, permetent càlculs ferms d'estructures, fent possible la minimització d'ús de materials i la viabilitat dels
        projectes.
        \item Generació de documentació tècnica en LaTex i manuals operatius per a operadors i mantenidors, assegurant la transferència de 
        coneixement dels sistemes de software i l'adequació per a futurs processos d'auditoria interna.
        \item Col·laboració estreta amb equips de software multidisciplinaris de Siemens Energy en tallers de millora contínua, fomentant la correcta integració de noves funcionalitats.
    \end{itemize}
    \vspace{0.3em}
    \textbf{Vestas Wind Systems A/S}, Software d'aplicació de Control de Turbines Eòliques,\\ 
    Novembre 2021 – Desembre 2023, Dinamarca (remot)
    \begin{itemize}
        \item Responsable del disseny, implementació i verificació de sistemes de lubricació i de gir de les pales de les turbines, 
        mitjançant tècniques de control electrònic, integrats en turbines eòliques per a parcs de generació d'energia renovable. 
        Desenvolupament amb C++ i Matlab/Simulink.
        \item Implementació de millores en la seguretat dels sistemes de lubricació, mitjançant el desenvolupament amb una metodologia 
        orientada en objectes d'un porotocol particular de comunicació i redundància entre dues bombes de lubricació.
        \item Lideratge en l'actualització i creació de tests, tant unitaris com d'integració, de les funcionalitats esmentades, aconseguint una 
        cobertura i un percentatge d'èxit del 100\% en la finalització del contracte.
        \item Participació en la creació de tests per a operadors de turbina, amb el corresponent disseny de UX i assegurant la integració
        del framework de test amb la lògica d'aplicació.
        \item Coordinació d'un equip de 20 desenvolupadors en la resolució de múltiples tasques de tots els models de turbina.
    \end{itemize}
\end{jobshort}


\begin{jobshort}{Enginyer de Software de Producte | Ficosa S.L (Sector Automoció, Barcelona)}{Febrer 2018 - Novembre 2021}
    Desenvolupament de software en C, a través de la metodologia SCRUM, per a sistemes d'assistència a l'aparcament pel Grup Volkswagen, 
    formant part de tota la cadena de desenvolupament (ASPICE), tant definició de requisits, el·laboració de l'arquitectura, 
    desenvolupament de software i desenvolupament de proves, tant unitàries com d'integració.
    En base els requisits, destacar la participació en equips multidisciplinaris per a la definició, execució i documentació del software,
    amb especial èmfasi en la traçabilitat de requisits i la mitigació de riscos relacionats amb la seguretat i la privacitat, 
    assegurant l'acompliment d'estàndards de la indústria (ISO 26262, AUTOSAR).
    
    Contribucions destacades:
    \vspace{0.3em}
    \begin{itemize}
        \item Lideratge tècnic en l'arquitectura de sistemes de càmeres (visió posterior i superior), assegurant la integració 
        segura i la compleció de tots els casos d'ús en referència a l'assistència d'aparcament, i la seva seguretat.
        \item Desenvolupament d'un algorisme de calibració de les càmeres en base als requisits funcionals.
        \item Depuració de problemes complexes a través de l'anàlisi de comunicacions en xarxes CAN, Ethernet i RTP, 
        amb detecció proactiva de vulnerabilitats o anomalies de seguretat.
        \item Desenvolpament (Python) d'una eina d'automatització de generació de tasques a Jira, des d'una plantilla d'Excel, 
        assegurant la traçabilitat de les tasques, agilitzant la gestió informàtica dels projectes un 90\%.
        \item Consecució de l'aprovació d'una auditoria interna on s'avaluava tot el procés de desenvolupament, des de requisits fins
        la validació, validant també la gestió del projecte i la seva documentació.
    \end{itemize}
\end{jobshort}

%Projects
\section{Projectes}

\textbf{2025 - Automatització de cerca de talents al "Futbol Fantasy"} \hfill \href{https://github.com/quimnaba/mister-md-scraper}{Mister MD Scraper} \\[3.75pt]
Desenvolupament d'un script de Python per identificar ràpidament 
"gangues" al mercat de "Mister Fantasy Football", específicament per a la primera divisió 
espanyola, a l'aplicació Mister Fantasy MD (BeManager).L'aplicació compara els jugadors 
disponibles al mercat amb una llista definida per uns experts, i l'script suggereix quins jugadors 
s'haurien de fitxar. L'objectiu del projecte és estalviar temps als usuaris optimitzant el 
procés de trobar jugadors de futbol valuosos, i poder competir i divertir-te amb els amics.\\

\textbf{2019 - Disseny i Implementació d'una xarxa d'Internet a l'Hospital Sant Joan de Déu de Lunsar, Sierra Leone} \hfill \\[3.75pt]
Participació en un projecte de digitalització, a través d'Aucop -associació de la UPC-, implementat a terreny, d'una xarxa   
d'Internet i Intranet, per tal de poder comunicar els servidors d'un hospital rural entre sí,
i tenir un punt d'accés a internet a través de 4G. La dificultat més gran va ser
haver de dibuixar el plànol de l'hospital in-situ (prèviament fet amb Google Maps) per tal de poder comprovar que 
el disseny estratègic de la xarxa era correcte i poder-lo optimitzar al màxim.\\

\textbf{2018 - Disseny i Implementació d'un carregador de bateries solars per equipamen mèdic a Gambo, Etiòpia} \hfill \\[3.75pt]
Participació en un projecte de digitalització, implementat a terreny, d'una xarxa   
d'Internet i Intranet, per tal de poder comunicar els servidors d'un hospital rural entre sí,
i tenir un punt d'accés a internet a través de 4G. La dificultat més gran va ser
haver de dibuixar el plànol de l'hospital in-situ (prèviament fet amb Google Maps) per tal de poder comprovar que 
el disseny estratègic de la xarxa era correcte i poder-lo optimitzar al màxim.\\

%----------------------------------------------------------------------------------------
%	EDUCATION
%----------------------------------------------------------------------------------------
\section{Educació}
2023 - Actualitat (Previst 2026)  \textbf{Màster d'Enginyeria Informàtica a Universitat Oberta de Catalunya} \\
Especialització: Sistemes de Software Avançats, Intel·ligència Artificial i Computació Distribuïda.\\
Compatibilitzant amb treball professional a temps complet.\\
Assignatures rellevants: Computació Avançada, Arquitectura al núvol i Sistemes Distribuïts, Intel·ligència Artificial i 
Ciberseguretat.\\

2013 - 2018 \textbf{Enginyeria de Telecomunicacions Universitat Politècnica de Catalunya}\\ 
Especialitat: Sistemes Electrònics.\\ 
Treball final sobre una eina d'automatització de PM a Ficosa.\\
Compatibilitzant els estudis amb l'execució dels projectes a l'Àfrica.

%----------------------------------------------------------------------------------------
%	PUBLICATIONS
%----------------------------------------------------------------------------------------
%\section{Publications}
%\begin{refsection}[citations.bib]
%\nocite{*}
%\printbibliography[heading=none]
%\end{refsection}

%----------------------------------------------------------------------------------------
%	SKILLS
%----------------------------------------------------------------------------------------
\section{Habilitats Professionals}
\begin{tabularx}{\linewidth}{@{}l X@{}}
    Generals &  \normalsize{Git, Windows, Linux, CAN, ETH, RTP, Android, Docker, Kubernetes}\\
    Programació &  \normalsize{Python, C, C++, Matlab, Simulink, Java, Bash, Visual Basic, XML, JSON, SQL, LaTex}\\
    Software Incrustat  &  \normalsize{Lauterbach TRACE32, J-Link Segger}\\  
    Sector eòlica &  \normalsize{Eines internes de configuració de la turbina, eines personalitzades de SAP}\\  
    Automoció &  \normalsize{Enterprise Architect, Eines de Vector (DaVinci Developer \& Configurator, P-CAN), ASPICE, DOORS}\\  
    Gestió de Projectes &  \normalsize{Jira, SVN, Eines de Microsoft Office (Excel, Word, PPT...)}  
\end{tabularx}

\section{Habilitats Personals}
\begin{tabularx}{\linewidth}{@{}l X@{}}
    Idiomes &  \normalsize{Català i Castellà (nadiu, C1 certificat) i Anglès (avançat, C1)}\\
    Generals &  \normalsize{Humilitat, Empatia i Escolta activa, Treball en equip, Lideratge i Presa de decisions, Pensament lògic, Tolerància i Respecte.}\\
    Interesos  &  \normalsize{Esports, Gimnàs, Videojocs, Fotografia, Viatges}\\  
    Carnets de Conduir &  \normalsize{A, A2, A1, AM, B i PER} 
\end{tabularx}

\vfill
\center{\footnotesize Última actualització: \today}

\end{document}
