%	The MIT License (MIT)
%
%	Copyright (c) 2021 Jitin Nair
%
%	Permission is hereby granted, free of charge, to any person obtaining a copy
%	of this software and associated documentation files (the "Software"), to deal
%	in the Software without restriction, including without limitation the rights
%	to use, copy, modify, merge, publish, distribute, sublicense, and/or sell
%	copies of the Software, and to permit persons to whom the Software is
%	furnished to do so, subject to the following conditions:
%	
%	THE SOFTWARE IS PROVIDED "AS IS", WITHOUT WARRANTY OF ANY KIND, EXPRESS OR
%	IMPLIED, INCLUDING BUT NOT LIMITED TO THE WARRANTIES OF MERCHANTABILITY,
%	FITNESS FOR A PARTICULAR PURPOSE AND NONINFRINGEMENT. IN NO EVENT SHALL THE
%	AUTHORS OR COPYRIGHT HOLDERS BE LIABLE FOR ANY CLAIM, DAMAGES OR OTHER
%	LIABILITY, WHETHER IN AN ACTION OF CONTRACT, TORT OR OTHERWISE, ARISING FROM,
%	OUT OF OR IN CONNECTION WITH THE SOFTWARE OR THE USE OR OTHER DEALINGS IN
%	THE SOFTWARE.
%	
%
%-----------------------------------------------------------------------------------------------------------------------------------------------%

%----------------------------------------------------------------------------------------
%	DOCUMENT DEFINITION
%----------------------------------------------------------------------------------------
\documentclass[a4paper,12pt]{article}

%----------------------------------------------------------------------------------------
%	FONT
%----------------------------------------------------------------------------------------

%----------------------------------------------------------------------------------------
%	PACKAGES
%----------------------------------------------------------------------------------------
\usepackage{url}
\usepackage{parskip} 	

%other packages for formatting
\RequirePackage{color}
\RequirePackage{graphicx} % <-- graphicx must be loaded before \icon definition
\usepackage[usenames,dvipsnames]{xcolor}
\usepackage[scale=0.9]{geometry}

%tabularx environment
\usepackage{tabularx}
\usepackage{array}

%for lists within experience section
\usepackage{enumitem}
\usepackage{needspace}

% centered version of 'X' col. type
\newcolumntype{C}{>{\centering\arraybackslash}X} 

%to prevent spillover of tabular into next pages
\usepackage{supertabular}
\usepackage{tabularx}
% \usepackage{ltablex}
\newlength{\fullcollw}
\setlength{\fullcollw}{0.47\textwidth}

%custom \section
\usepackage{titlesec}				
\usepackage{multicol}
\usepackage{multirow}
\usepackage[catalan]{babel}
\usepackage{fontawesome5} % Paquet d'icones modern
\definecolor{linkcolour}{rgb}{0,0.2,0.6}
\definecolor{sectioncolor}{rgb}{0.2,0.4,0.7}
    
%CV Sections inspired by: 
%http://stefano.italians.nl/archives/26
\titleformat{\section}{\large\scshape\raggedright\color{sectioncolor}}{}{0em}{}[\color{sectioncolor}\titlerule]
\titlespacing{\section}{0pt}{12pt}{6pt}

%for publications
\usepackage[style=authoryear,sorting=ynt, maxbibnames=2]{biblatex}

%Setup hyperref package, and colours for links
\usepackage[unicode, draft=false]{hyperref}
\definecolor{linkcolour}{rgb}{0,0.2,0.6}
\hypersetup{colorlinks,breaklinks,urlcolor=linkcolour,linkcolor=linkcolour}
% Make PDF strings safe: remove/replace fragile/icon commands when hyperref writes bookmarks
\pdfstringdefDisableCommands{%
    % remove visual icon in bookmarks
    \def\icon#1{}%
    % replace fontawesome macros with plain text labels for bookmarks
    \def\faUser{Perfil}%
    \def\faBriefcase{Experi\`encia}%
    \def\faProjectDiagram{Projectes}%
    \def\faGraduationCap{Educaci\'o}%
    \def\faCogs{Habilitats}%
    \def\faUserFriends{Habilitats Personals}%
    \def\faLock{Seguretat}%
    \def\faLinkedin{LinkedIn}%
    \def\faEnvelope{Email}%
    \def\faMobile{Tel\`efon}%
    \def\faGithub{GitHub}%
    \def\faHome{Localitat}%
    \def\faSyncAlt{Actualitzaci\'o}%
}
\addbibresource{citations.bib}
\setlength\bibitemsep{1em}

%for social icons
\usepackage{fontawesome5}

%debug page outer frames
%\usepackage{showframe}

% Custom commands for icons (make robust so it won't break in moving arguments)
% graphicx must be loaded before this command
\DeclareRobustCommand{\icon}[1]{\raisebox{-0.1\height}{\color{sectioncolor}#1}}

% job listing environments
% joblong: Title (bold) on first line, date (small italic) left-aligned on next line, then itemize body
\newenvironment{joblong}[2]
    {
    \vspace{4pt}
    \noindent\textbf{#1}\par
    \noindent{\small\itshape #2}\par
    \begin{minipage}[t]{\linewidth}
    \begin{itemize}[nosep,after=\strut, leftmargin=1em, itemsep=3pt,label=--]
    }
    {
    \end{itemize}
    \end{minipage}
    \vspace{6pt}
    }

% jobshort: same format but slightly smaller title font and no automatic itemize wrapper
\newenvironment{jobshort}[2]
    {
    \vspace{4pt}
    \noindent{\small\textbf{#1}}\par
    \noindent{\small\itshape #2}\par
    }
    {
    \vspace{6pt}
    }


%----------------------------------------------------------------------------------------
%	BEGIN DOCUMENT
%----------------------------------------------------------------------------------------
\begin{document}

% non-numbered pages
\pagestyle{empty} 

%----------------------------------------------------------------------------------------
%	TITLE
%----------------------------------------------------------------------------------------

\begin{tabularx}{\linewidth}{@{}l X@{}}
\begin{minipage}[c]{0.20\linewidth}
\includegraphics[width=2.6cm,height=2.6cm,keepaspectratio]{media/ProfilePicture.png}
\end{minipage} &
\begin{minipage}[c]{0.78\linewidth}
{\Huge\bfseries Quim Nadal Bargués}\\[4pt]
{\small
\href{https://www.linkedin.com/in/quimnadalbargues/}{\color{linkcolour}\faLinkedin\ \ Quim Nadal Bargués} \ $|$ 
\href{mailto:quimnaba@gmail.com}{\color{linkcolour}\faEnvelope\ \ quimnaba@gmail.com} \ $|$ \\
\href{tel:+34620200817}{\color{linkcolour}\faMobile\ \ +34 620.200.817} \ $|$ 
\href{https://github.com/quimnaba123}{\color{linkcolour}\faGithub\ \ quimnaba123} \ $|$ 
\href{}{\raisebox{-0.05\height}{\color{linkcolour}\faHome} \textcolor{linkcolour}{Barcelona}}
}
\end{minipage} \\
\end{tabularx}
\vspace{6pt}
\noindent{\color{linkcolour}\rule{\linewidth}{1pt}}\par\vspace{6pt}

%----------------------------------------------------------------------------------------
% EXPERIENCE SECTIONS
%----------------------------------------------------------------------------------------

%Interests/ Keywords/ Summary
\section{\texorpdfstring{\icon{\faUser}\ Perfil Professional}{Perfil Professional}}
Enginyer de Telecomunicacions amb més de 7 anys d'experiència en entorns d'alt rendiment i alta disponibilitat, especialitzat en l'arquitectura, 
gestió i seguretat d'infraestructures tecnològiques complexes. Sòlida trajectòria liderant projectes crítics, definint estratègies tècniques i assegurant 
la resiliència, el rendiment i la seguretat dels sistemes. Experiència en governança TIC, disseny de xarxes, supervisió d'entorns híbrids (on-premise/cloud), 
i implementació de polítiques de seguretat per garantir la integritat, confidencialitat i disponibilitat de la informació. Capacitat demostrada per dirigir equips tècnics, 
optimitzar processos operatius i alinear la infraestructura tecnològica amb els objectius del negoci.

%Experience
\section{\texorpdfstring{\icon{\faBriefcase}\ Experiència Professional}{Experiència Professional}}

\begin{itemize}[leftmargin=*,label=\textbullet]
\item \begin{jobshort}{\includegraphics[width=1.2cm]{media/DisCreadis.png} 
    	\textbf{\large Enginyer de Productes de Software | Creadis S.A}}{Novembre 2021 - Actualitat, Barcelona (Híbrid)}
    \textbf{Responsable tècnic de l'arquitectura, seguretat i disponibilitat dels sistemes de control} 
    per a infraestructures crítiques de generació i transport d'energia. 
    Direcció i implementació del cicle de vida complet dels sistemes, amb especial atenció als requisits de \textbf{ciberseguretat, redundància, monitorització i continuïtat operativa}. 
    Lideratge en la definició d'estratègies tècniques i avaluació de riscos per garantir la fiabilitat dels serveis.
    A través de Creadis, he col·laborat amb empreses líders del sector energètic a Dinamarca, com Vestas i Siemens Energy, 
    aportant solucions innovadores i robustes per a la gestió de sistemes crítics.
    En concret, les meves contribucions inclouen:\\[0.6em]

    {\large\textbf{\raisebox{-0.12\height}{\includegraphics[width=1.0cm]{media/Vestas.jpeg}} Vestas Wind Systems A/S, Estratègia i Governança per a un Sistema de Control de Planta}} \\
    \noindent{\small\itshape Setembre 2025 - Actualitat, Dinamarca (remot)}\par
    \begin{itemize}
        \item \textbf{Definició de l'estratègia tècnica} i l'arquitectura de sistema per al controlador d'una planta 
        eòlica, assegurant l'escalabilitat, el manteniment i la seguretat del sistema.
        \item \textbf{Gestió de riscos i governança:} Avaluació exhaustiva dels riscos tècnics i de seguretat. 
        Definició i aplicació de polítiques i estàndards de desenvolupament segur.
        \item \textbf{Supervisió de l'arquitectura:} Disseny modular i resilient per garantir l'alta disponibilitat 
        i facilitar el manteniment futur de la infraestructura de control.
    \end{itemize}

    \vspace{0.6em}
    \Needspace{6\baselineskip}
    {\large\textbf{\raisebox{-0.12\height}{\includegraphics[width=1.0cm]{media/siemensEnergy.png}}\linebreak[2] 
    Siemens Energy, SA, Infraestructura de Software i Gestió de Configuracions}}\\
    \noindent{\small\itshape Gener 2023 - Juliol 2025, Dinamarca (remot)}\par
    \begin{itemize}
        \item \textbf{Supervisió de la integritat del sistema:} Responsable de l'arquitectura de software que processa 
        dades de centenars de sensors, garantint la fiabilitat i disponibilitat dels càlculs estructurals crítics.
        \item \textbf{Governança documental:} Establiment i manteniment d'un sistema integral de documentació tècnica 
        (LaTeX), assegurant el compliment d'estàndards de qualitat i preparant els sistemes per a auditories.
        \item \textbf{Coordinació tècnica:} Col·laboració estreta amb equips multidisciplinaris per a la integració 
        segura de noves funcionalitats en la infraestructura existent.
    \end{itemize}

    \vspace{0.6em}
    {\large\textbf{\raisebox{-0.12\height}{\includegraphics[width=1.0cm]{media/Vestas.jpeg}} Vestas Wind Systems A/S, Seguretat, Redundància i Monitorització de Sistemes}} \\
    \noindent{\small\itshape Novembre 2021 - Desembre 2023, Dinamarca (remot)}\par
    \begin{itemize}
        \item \textbf{Arquitectura de sistemes resilients:} Disseny i implementació de sistemes de control amb protocols 
        de comunicació redundants per a sistemes crítics (lubricació, gir de pales), augmentant significativament la 
        disponibilitat.
        \item \textbf{Estratègia de seguretat aplicada:} Desenvolupament d'un protocol de comunicació segur i robust entre 
        components crítics, orientat a objectius d'integritat i confidencialitat.
        \item \textbf{Gestió integral de la infraestructura de proves:} Lideratge en la creació i manteniment d'un entorn 
        de proves (unitari i d'integració) amb cobertura del 100\%, essencial per a la fiabilitat i el rendiment dels serveis.
        \item \textbf{Supervisió d'equip:} Coordinació tècnica d'un equip de 20 desenvolupadors, assegurant l'alineació 
        amb les estratègies de qualitat i seguretat definides.
    \end{itemize}
\end{jobshort}\item \begin{jobshort}{\includegraphics[width=1.2cm]{media/Ficosa.jpg} \textbf{\large Enginyer de Software de Producte | Ficosa S.L (Sector Automoció, Barcelona)}}{Febrer 2018 - Novembre 2021}
    Desenvolupament de software segur per a sistemes crítics d'assistència a la conducció, amb estrictes requisits de 
    disponibilitat i seguretat (ISO 26262, AUTOSAR). Experiència en la governança de processos de desenvolupament 
    (ASPICE) i en l'assegurament de la qualitat i traçabilitat en tot el cicle de vida del software.
    
    Contribucions destacades:
    \vspace{0.3em}
    \begin{itemize}
        \item \textbf{Arquitectura de xarxes i comunicacions segures:} Lideratge en el disseny i integració de sistemes 
        de càmeres, assegurant la comunicació fiable i segura a través de busos CAN, Ethernet i protocols RTP. 
        Detecció i anàlisi de vulnerabilitats en les comunicacions.
        \item \textbf{Governança del cicle de vida:} Aprovació d'auditories internes que avaluaven tot el procés de 
        desenvolupament (requisits, disseny, implementació, validació), 
        garantint el compliment dels estàndards de la indústria.
        \item \textbf{Automatització i eficiència operativa:} Desenvolupament d'una eina interna (Python) per a 
        l'automatització de la gestió de projectes (Jira), millorant la 
        traçabilitat i optimitzant els processos un 90\%.
    \end{itemize}
\end{jobshort}

\item \begin{jobshort}{\includegraphics[width=1cm]{media/itnow.jpeg} \textbf{\large Pràctiques de Consultoria Tecnològica | ITNow (Sector Banca, Barcelona)}}{Setembre 2017 - Gener 2018}
    Primera experiència professional en entorns de sistemes i infraestructures, participant en projectes de gestió de xarxes i 
    administració de sistemes per a grans entitats financeres.
    
    Tasques i aprenentatges clau:
    \begin{itemize}
        \item \textbf{Gestió d'infraestructura IP:} Participació en el desenvolupament i implementació d'una aplicació per a 
        la gestió centralitzada del pla d'adreces IP de \textbf{La Caixa} i proveïdors externs.
        \item \textbf{Administració de servidors web:} Aprenentatge i aplicació bàsica de configuracions en entorns \textbf{Apache}, incloent-hi la instal·lació 
        i gestió de servidors web i la implementació de protocols de seguretat \textbf{SSL/TLS} per a comunicacions xifrades.
        \item \textbf{Bases de dades i operacions:} Experiència inicial en la configuració i administració de bases de dades \textbf{MySQL}, 
        centrant-se en el disseny de schemas i consultes per a aplicacions de gestió.
        \item \textbf{Virtualització i infraestructura:} \textbf{Aprenentatge profund en la operació i gestió de màquines virtuals}, incloent-hi la creació, 
        configuració, clonació i manteniment d'entorns virtualitzats.
    \end{itemize}
\end{jobshort}
\end{itemize}

%Projects
\section{\texorpdfstring{\icon{\faProjectDiagram}\ Projectes Rellevants}{Projectes Rellevants}}
\begin{tabularx}{\linewidth}{@{}X r@{}}
 {\large\textbf{\includegraphics[width=0.6cm]{media/mister.png} 2025 - Arquitectura i Automatització d'un Sistema de Procesament de Dades}} & \\
 & \vspace{4pt} \href{https://github.com/quimnaba123/mister-md-scraper}{\faGithub\ Mister MD Scraper} \\[3.75pt]
\end{tabularx}
    Disseny i implementació d'un \textbf{sistema automatitzat, paral·lel} (Python) per al recopilació, processament i 
    anàlisi de dades a gran escala. El sistema pren dades de diverses fonts, les compara amb regles de negoci 
    definides i genera recomanacions accionables. 
    Il·lustra la capacitat per crear eines que optimitzen processos, estalvien temps i donen suport a 
    la presa de decisions.\\

{\large\textbf{\includegraphics[width=0.8cm]{media/stjohngod.jpeg} 2019 - Disseny, Desplegament i Administració d'una Xarxa Corporativa}} \hfill \\[3.75pt]
    \textbf{Lideratge en el disseny i implementació in-situ} de la infraestructura de xarxa (LAN/WAN) per a un hospital rural. 
    Responsable de tot el cicle: 
    Definició del projecte col·laborant estretament amb el personal sanitari de l'Hospital Sant Joan de Déu, de Barcelona,
    avaluació de les necessitats de la contrapart, disseny estratègic del pla de xarxa (optimitzant cobertura i rendiment), 
    instal·lació física (cablejat, punt d'accés 4G), 
    configuració de servidors i commutadors, i assegurament de la connectivitat i disponibilitat dels serveis crítics 
    (internet i intranet).\\

{\large\textbf{\includegraphics[width=1cm]{media/Gambo.png} 2018 - Disseny i Implementació d'un carregador de bateries solars per equipamen mèdic a Gambo, Etiòpia}} \hfill \\[3.75pt]
    \textbf{Lideratge en el disseny i la implementació d'un projecte de digitalització}, 
    col·laborant estretament amb la universitat ESADE, i 
    la fundació Iñaki Alegría, dins del marc de l'associació Aucop -associació de la UPC-,   
    per la digitalització d'un centre mèdic rural a Gambo, Etiòpia. 
    Es va dur a terme el disseny i la implementació d'un carregador de bateries solars
    per equipamen mèdic, per tal de garantir el subministrament elèctric necessari 
    per tal de desacoblar-se de les interrupcions constants del subministrament elèctric de la zona.\\

%----------------------------------------------------------------------------------------
%	EDUCATION
%----------------------------------------------------------------------------------------
\Needspace{20\baselineskip}
\section{\texorpdfstring{\icon{\faGraduationCap}\ Educació}{Educació}}
\begin{tabularx}{\linewidth}{@{}X >{\raggedleft\arraybackslash}p{0.28\linewidth}@{}}
% UOC entry: left = content, right = date (italic, small)
{\raisebox{-0.65\height}{\includegraphics[width=0.9cm]{media/uoc.png}}\hspace{0.5em}
\begin{minipage}[t]{\linewidth}
	\textbf{Màster d'Enginyeria Informàtica}\\
	\textbf{Universitat Oberta de Catalunya}\\
    Especialització: Sistemes de Software Avançats, Intel·ligència Artificial i Computació Distribuïda.\\
    \textbf{Assignatures directament aplicables:} \textit{Arquitectura al núvol i Sistemes Distribuïts} 
    (disseny d'infraestructures escalables), \textit{Ciberseguretat} (governança, polítiques, anàlisi de riscos), 
    \textit{Computació Avançada} (rendiment i optimització).
\end{minipage}}
& {\begin{minipage}[t]{\linewidth}\footnotesize\itshape 2023 - Actualitat (Previst 2026)\end{minipage}} \\
\\[6pt]
% UPC entry
{\raisebox{-0.65\height}{\includegraphics[width=0.9cm]{media/upc.png}}\hspace{0.5em}
\begin{minipage}[t]{\linewidth}
	\textbf{Enginyeria de Telecomunicacions}\\
	\textbf{Universitat Politècnica de Catalunya}\\
    Especialitat: Sistemes Electrònics (fonaments de xarxes, comunicacions, sistemes digitals).\\
    Etapa on es van implementar els projectes rellevants a l'Àfrica.
\end{minipage}}
& {\begin{minipage}[t]{\linewidth}\footnotesize\itshape 2013 - 2018\end{minipage}} 
\end{tabularx}

%----------------------------------------------------------------------------------------
%	PUBLICATIONS
%----------------------------------------------------------------------------------------
%\section{Publications}
%\begin{refsection}[citations.bib]
%\nocite{*}
%\printbibliography[heading=none]
%\end{refsection}

%----------------------------------------------------------------------------------------
%	SKILLS
%----------------------------------------------------------------------------------------
\section{\texorpdfstring{\icon{\faCogs}\ Habilitats Professionals}{Habilitats Professionals}}
\begin{tabularx}{\linewidth}{@{}l X@{}}
    \icon{\faNetworkWired} \textbf{Infraestructura \& Xarxes} &  Arquitectura de Sistemes, Xarxes LAN/WAN (disseny i diagnòstic), Comunicacions (CAN, ETH, RTP), Virtualització (Docker/Kubernetes *), Linux/Windows Server, Principis de Cloud Híbrid.\\
    \icon{\faLock} \textbf{Seguretat \& Governança} &  Ciberseguretat en Sistemes Crítics, Anàlisi de Riscos, Compliment d'Estandards (ISO 26262, ASPICE), Polítiques de Seguretat, Auditoria Interna, Documentació Tècnica (LaTeX).\\
    \icon{\faCode} \textbf{Desenvolupament \& Automatització} &  Python (Scripting, Eines Internes), C/C++, Matlab/Simulink, Bash, Git, SQL.\\
    \icon{\faTasks} \textbf{Gestió \& Operacions} &  Supervisió d'Equips Tècnics, Gestió de Projectes (Jira, Metodologies Àgils/SCRUM), Monitorització de Sistemes, Manteniment Preventiu/Correctiu, Continuïtat del Negoci.\\
\end{tabularx}
\vspace{0.2em}
{\footnotesize \textit{* Nota: Experiència teòrica i acadèmica en Docker/Kubernetes, amb forts fonaments en computació distribuïda i containers, en procés d'implementació pràctica en projectes actuals.}}

\Needspace{10\baselineskip}
\section{\texorpdfstring{\icon{\faUserFriends}\ Habilitats Personals}{Habilitats Personals}}
\begin{tabularx}{\linewidth}{@{}l X@{}}
    \icon{\faLanguage} Idiomes &  Català i Castellà (nadiu, C1 certificat) i Anglès (avançat, C1). 
    Alt grau d'experiència amb relacions internacionals.\\
    \icon{\faChessKing} Lideratge &  Capacitat per dirigir equips tècnics, prendre decisions estratègiques amb visió global i motivar per assolir els objectius de seguretat.\\
    \icon{\faCogs} Pensament Sistèmic &  Capacitat per analitzar i dissenyar sistemes complexos, entenent les interdependències entre infraestructura, seguretat i negoci.\\
    \icon{\faHandshake} Comunicació &  Empatia, escolta activa i capacitat per explicar conceptes complexos a audiències tècniques i executives. 
    Capacitat de treball en entorns multidisciplinaris.
\end{tabularx}

\vfill
\begin{center}
    {\footnotesize \icon{\faSync} Última actualització: \today}
\end{center}
\end{document}